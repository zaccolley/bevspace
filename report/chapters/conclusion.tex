\chapter{Conclusion} \label{c}

\section{Introduction} \label{c--introduction}

In this Chapter, we first summarise the work described in this report (section x). Then we draw a number of conclusions about key parts of the work undertaken in section y, and finally in section z we discuss future work and how we see Semantic Web technologies helping support projects such as this one.

\section{Summary} \label{c--summary}

This is a summary of each chapter  intro and summary
Chapter 1 introduced blah.
Chapter 2 reviewed the state-of-the-art in blah and blah.  Blah was introduced and blah described. The potential for blah blah was highlighted.
Chapter 3 describes the design of blah blah. The separate functions of blah blah that support the requirements were then described in more detail, including blah and blah.
Chapter 4 described the implementation blah blah.
Chapter 5 presented a series of tests that demonstrate blah blah.

\section{Conclusions} \label{c--conclusions}

The aim of this project was to blah blah.  We chose to focus on blah blah.
We then designed and implemented a system that could:
blah
blah
blah
blah
These combined capabilities blah blah blah.
In Chapter 1 we state the general hypothesis that blah blah blah. We have tested this thesis by blah blah.

\section{Future Work} \label{c--future-work}

% for this project, whatever i didnt finish lol
% recipe editing etc

% \subsection{User Interface}

% push notifications forced the usage of a microservice server due to the GCM
% When registering a user for push notifications there are some best practices in terms of UX.

% Signing up a user with no context is strongly discouraged, the user should always know what they've signed up for through an action. Giving the user even more control by being able to toggle the notifications after initially registering is recommended. \cite{best_practises_push_notifications}


% long term:
% physical web example

% integration with physical systems
