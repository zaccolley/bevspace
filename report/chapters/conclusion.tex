\chapter{Conclusion} \label{c}

\section{Introduction} \label{c--introduction}

In this chapter, the project is summarised overall. Then some conclusions are drawn about the different parts of the work in section \ref{c--summary} and finally in section \ref{c--future-work} future work is discussed.

\section{Summary} \label{c--summary}

Chapter \ref{i} introduced the project defining project goals and the motivation for the project. It also gave an outline for how the report was structured.

Chapter \ref{l-r} reviewed user experience through performance and an overview of home-brewing. Some background and software surrounding home-brewing was shown. Techniques, principles and examples of performance were then explained.

Chapter \ref{a-d} described the proposed solution design. Tooling, methodologists, choice of technologies and data were discussed in this chapter. This is also where the requirements were made.

Chapter \ref{s-i} described the system implementation of the proposed solution and testing. In this chapter, there was discussion interesting problems and the implementation overall.

Chapter \ref{t-e} described the testing and evaluation. This section explained the set-up of testing, the results found from testing as well as evaluation of the project as a whole.

\section{Conclusions} \label{c--conclusions}

The aim of this project was to create progressive web application. The project made great introductions to many technologies to achieve this goal such as service workers for caching, offline features through synced clientside databases and performance in general.

A lot was learnt about authoring a application in this area and this project gave a starting point to many avenues of performance related development and discussion.

Overall it was found that developing with this feature set is challenging. Building a progressive web application in the current web environment is interesting with definite room to make impact in how technologies are shaped.

With technologies such as service worker being pushed by engineers rather than standard bodies there is a lot of progress, however there is a definite need for my simpler developer ergonomic implementations of service worker related features.

\section{Future Work} \label{c--future-work}

Initially the requirements not fulfilled in the implementation would be the first to be considered as future work. Bringing in more complex functionality as seen in other software such as recipe editing, social network features could be explored too.

Moving away from the core concept of this project looking at integration with brewing set-ups and allowing for automation would be a great direction to take. Especially looking at projects such as the Physical Web which compliments progressive web applications. \cite{physical_web}

\section{Closing words} \label{c--closing-words}

The web is a place of ever changing challenge. For the forthcoming future, progressive web applications show a great promise to bring more accessibility to people everywhere.
