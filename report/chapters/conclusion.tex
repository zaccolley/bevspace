\chapter{Conclusion} \label{c}

\section{Introduction} \label{c--introduction}

In this chapter, the project is summarised overall. Then some conclusions are drawn about the different parts of the work in section \ref{c--summary} and finally in section \ref{c--future-work} future work is discussed.

\section{Summary} \label{c--summary}

Chapter 1 reviews home-brewing and user experience through performance.

Chapter 2 describes the design of the proposed solution.

Chapter 3 describes the system implementation of the proposed solution and testing.

Chapter 4 describes the testing and evaluation.

Chapter 5 is the conclusion with future work.

\section{Conclusions} \label{c--conclusions}

The aim of this project was to blah blah.  We chose to focus on blah blah.
We then designed and implemented a system that could:
blah
blah
blah
blah
These combined capabilities blah blah blah.
In Chapter 1 we state the general hypothesis that blah blah blah. We have tested this thesis by blah blah.

\section{Future Work} \label{c--future-work}

% for this project, whatever i didnt finish lol
% recipe editing etc

% push notifications forced the usage of a microservice server due to the GCM
% When registering a user for push notifications there are some best practices in terms of UX.

% Signing up a user with no context is strongly discouraged, the user should always know what they've signed up for through an action. Giving the user even more control by being able to toggle the notifications after initially registering is recommended. \cite{best_practises_push_notifications}

% long term:
% physical web example

% integration with physical systems
