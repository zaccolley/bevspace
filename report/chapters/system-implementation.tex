\chapter{System Implementation}

\section{Introduction} \label{s-i--introduction}

This chapter described the high-level requirements and design of a system that blah blah blah.  The chapter started by describing blah.  The proposed solution was then discussed in section blah followed by blah in section blah, etc.
Blah blah is covered in further detail in Chapter 4 which describes the implementation of blah blah.

\section{Blah blah}

\section{Interesting Problems} \label{s-i--interesting-problems}

During the project implementation several issues were identified that merit discussion. This section addresses those topics, which are blah, blah and blah.

\section{Summary} \label{s-i--summary}

This chapter described the implementation of blah blah, which was based on the design described in Chapter 3. The implementation was introduced in section x and blah blah blah. Once the general implementation details had been introduced, several interesting implementation problems were addressed in section q, including the detailed coverage of blah blah.

\section{Testing and Evaluation} \label{s-i--testing-and-evaluation}

\subsection{Introduction}

Chapters 3 and 4 described the design and implementation of blah blah, a system that blah blah blah.  In this chapter, we present a testing method and its results that show blah blah blah.  The chapter is organised as follows:  Section x introduces blah and describes blah.  Next, section y presents blah, etc.
The results of the tests are summarised in section z, before the solution is evaluated in section qqq.
Write what you did, why you did it and how you did it here.

\subsection{Testing Summary}

In total over n tests were executed. Each test was blah blah blah, and this data was then used to blah. The tests illustrate that blah blah. In the next section, this is evaluated and the extent to which it supports the thesis is discussed.

\section{Evaluation}


When registering a user for push notifications there are some best practises in terms of UX.

Signing up a user with no context is sttrongly discouraged, the user should always know what they've signed up for through an action. Giving the user even more control by being able to toggle the notifications after initially registering is recommended. \cite{best_practises_push_notifications}

Chapter 1 highlighted the problem of blah blah blah. Chapter 2 reviewed the state-of-the-art in blah and blah.  Chapter 3 identified a set of technical requirements underpinning the development of blah, and the implementation of a blah blah was described in Chapter 4.

The testing described in section x demonstrates that blah blah blah. In this section therefore, we evaluate the implementation and discuss issues in the underlying technologies that the implementation has highlighted.

    \subsection{Requirements Review} \label{s-i--requirements-review}

    This section reviews the implemented platform, referring back to the requirements to identify the extent to which each has been fulfilled, and reflecting on their relevance for future work. Each of the requirements is reintroduced and discussed in turn.
    Refer back to each specified requirement and discuss...


    \subsection{Artefact Review} \label{s-i--artefact-review}

    This section reviews the implemented platform, referring back to the requirements to identify the extent to which each has been fulfilled, and reflecting on their relevance for future work. Each of the requirements is reintroduced and discussed in turn.
    Refer back to each specified requirement and discuss...

\section{Testing and Evaluation Summary}

This chapter introduced blah and blah.  In section x a series of tests were described which demonstrated blah blah.
An evaluation of blah blah was then presented in section y.  Section z revisited the requirements described in Chapter 3 and identified that blah blah blah. Finally in section q the aspects of blah blah were discussed.
